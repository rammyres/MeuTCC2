\chapter{Justificativa}
Desprende-se da análise da legislação vigente em relação as Eleições no Brasil a necessidade da manutenção de dois conceitos aparentemente antagônicos entre si: privacidade e publicidade; enquanto é preponderante para manutenção da liberdade de escolha do eleitor, é necessário que o eleitor seja identificado através de dados públicos e a divulgação irrestrita dos resultados das votações. 

A utilização do atual protocolo eleitoral, muito embora preveja mecanismos para garantir que o voto não será associado a seu emissor e haja ampla divulgação dos resultados das votações, a ausência de transparência em várias das etapa não deixa claro como essas proteções são executadas, não parecendo que sejam suficientes para atender aos critérios acima. Os \textit{softwares} e produtos de votação são sigilosos e não há como analisá-los. 

Tecnologias emergentes, como as apresentadas neste trabalho, entretanto são fortemente amparadas em privacidade e publicidade simultâneas. Os protocolos utilizados habitualmente enfocam na manutenção do segredo das identidades das partes que realizam as transações, entretanto todos os registros dessas transações são públicos e acessíveis a todos os usuários da rede, inclusive para os que não fizeram parte delas. A publicidade é tão relevante, que só quando a maioria dos nós da rede podem ter acesso e validar os dados de certa transação é que ela se torna válida, preceito conhecido como consenso, não podendo mais ser revertida. 

Assim sendo nos parece que a utilização dessas tecnologias, com as adaptações necessárias ao atendimento de certas questões legais, é uma alternativa viável para criação de um sistema eleitoral que agregue as características essenciais aqui discutidas. 