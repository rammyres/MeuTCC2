\begin{resumo}
	%Resumo
	\vspace{\onelineskip}
	\noindent
	O objetivo deste trabalho é discutir as tecnologias de votação eletrônica, além de apresentar uma nova alternativa, baseada na tecnologia blockchain, o Registro Distribuído de Votação Eletrônica, bem como discutir seus resultados. Historicamente votar é parte do cotidiano das democracias e com o crescimento delas, organizar o processo de votação e apurar os resultados se tornou cada vez mais desafiador. Por causa disso mecanismos eletrônicos e automáticos para apuração eleitoral se tornaram comuns. O Brasil adotou seu sistema de votação eletrônico na década de 90, entretanto o mesmo tem sido criticado por falhas e falta de transparência. Em 2008 a tecnologia blockchain foi exposta ao mundo, permitindo sistemas transparentes, mas que preservam a privacidade, despertando a atenção de especialistas, que tem se debruçado para criar um sistema de votação baseado em blockchain. 
	Para atender propósito o RDVE será desenvolvido e testado através de eleições simuladas e coleta dos produtos das votações. Como resultado pretendemos avaliar o \textit{software} proposto e analisar sua adequação para atendimento aos quesitos de confidencialidade, confiabilidade e auditabilidade.
	\par\textbf{Palavras-chave}: eleição, votação, eletrônica, digital, blockchain, registro. 
\end{resumo}


