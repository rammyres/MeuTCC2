\chapter{Discussão}

Levando em conta a tendência nas pesquisas mundiais e os resultados dos testes deste trabalho, a possibilidade de utilização de \textit{blockchain} para a construção de sistemas de votação, dotados de confidencialidade, confiabilidade e auditabilidade, parece viável.

Todos os dados produzidos nos aplicativos aqui descritos são persistidos como arquivos ou dados \gls{json}, meio legível não só para computadores, mas facilmente acessíveis para humanos, o que permite um elevado grau de auditabilidade. O padrão é aberto e tem suporte em virtualmente todas as linguagens modernas, incluindo as duas principais adotadas neste trabalho, Dart e Python, o que permitiria o desenvolvimento de ferramentas para auditar grandes volumes de dados. Os votos, especificamente, também são embaralhados o suficiente para impedir a recuperação da ordem de entrada na urna. Por fim, considerando princípios gerais da criptografia, como a impossibilidade de reversão e negação da origem, pelo uso de assinaturas criptográficas na assinatura do alistamento de eleitores, registro de candidatos, requisições de votação, cédulas, registros e produtos de votação, permitem o rastreio completo de todo o processo de votação, inspirando grande grau e confiabilidade.

Muito embora ainda esteja em fases iniciais e maiores pesquisas se mostrem necessárias, inclusive com envolvimento de mais partes e ampliação dos sistemas para abranger modelos mais complexos de votação, como votações para cargos múltiplos, nos parece que o refinamento do projeto RDVE, como um todo, demonstra viabilidade.