\chapter{Problematização}
Questões relacionadas à segurança e privacidade do modelo de votação brasileiro, especialmente a Urna Eletrônica Brasileira, tem sido recorrentes. Em seu livro O Mito da Urna o professor da UFMG, Dr. Jeroen Van de \citeonline{VandeGraaf2017}, explora diversas falhas no nosso modelo de urna eletrônica. Falhas atribuídas a possibilidade de violação de privacidade, violação da integridade dos votos ou recuperação da ordem de votação são discutidas extensivamente no trabalho.  

O professor Dr. Van de Graaf também cita trabalhos do Dr. Diego Aranha, que dedicou atenção especial a falhas contidas na urna eletrônica brasileira, como seu artigo apresentado no SBSeg \cite{Aranha2018}, em que detalha a possibilidade de execução de código arbitrário na urna e seu relatório produzido com base na análise direta do software da urna, “Vulnerabilidades no \textit{software} da urna eletrônica brasileira” \cite{AranhaKMS14}, há menção de erros preocupantes, como a possibilidade de recuperação da ordem dos votantes, implicando em violação de sigilo de voto, além de demonstração de que o processo de cifragem é inadequado e o algoritmo obsoleto. 

Diversas são as críticas traçadas pelos especialistas citados, e muitas são replicadas pelo público geral,  que se foca especialmente na falta de transparência do processo e ausência de auditoria pública. 

O que se questiona neste trabalho é a possibilidade de utilização das tecnologias discutidas, especialmente o Registro Distribuído, para conseguir sanar os problemas levantados, especialmente a ausência de transparência e os questionamentos quanto a privacidade do voto.  