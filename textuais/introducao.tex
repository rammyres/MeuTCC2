\chapter{Introdução}
Votar é um princípio básico das democracias, das antigas até as modernas e tem sido associado a capacidade popular de tomar decisões e governarem as si próprias. Registros de sistemas de votação datam desde o século VI a.c. \cite{BLACKWELL2003}, quando foi introduzido o modelo ateniense de democracia. Os gregos votavam através dos \textit{ostracon}, fragmentos de cerâmica que retinham, por exemplo, o nome dos cidadãos da \textit{\Gls{polis}} que poderiam ser banidos (ostracismo).  

Ao longo da história o modelo ateniense foi incorporado por outras culturas, como a romana, que adaptaram o conceito de votação através das \textit{leges tabellariae} (\textit{garbinia}, \textit{cassia}, \textit{papiria} e \textit{caelia}) nas decisões populares nos \textit{\gls{comitia}} e eleições de magistrados durante o século II a.c. \cite{Yakobson1995} e avançou até as democracias modernas, que incorporaram conceitos de segredo e privacidade, principalmente através do chamado Voto Australiano \cite{Newman2003} ainda no século XIX.  

O aumento do número de votantes, no entanto, passou a tornar-se um problema. Para exemplificar, o número de votantes nos Estados Unidos em eleições presidenciais subiu de menos de 75.000 em 1800 \cite{VoteArchive2015} para mais quase 136.000.000 em 2016 \cite{VoteArchive2016}. Essa evolução no ínterim exigiu mudanças de estratégias, já que o sistema de contagem voto a voto mostrou-se ineficiente na contagem de tal volume de votos \cite{BATTAGLINI2007}. 

Ainda durante o século XIX as primeiras tentativas de criação de meios automatizados de votação e contagem foram criados, como a “Máquina de Votar Elétrica” de \citeonline{wood_1898}. Durante o século XX muitas tentativas foram abordadas, com o virtual monopólio dos modelos de votação do AVM Corporation nos anos 60 a 80 como destaque \cite{Jones2012}. O Brasil começou a ponderar a utilização de modelos eletrônicos a partir da década de 80 \cite{Brasil2014}, passando a implantar o atual modelo durante a década de 90.  

O modelo adotado por aqui, no entanto, não se livrou de críticas. Durante a primeira e segunda décadas dos anos 2000 problemas como violação do sigilo do voto e execução de código arbitrário puderam ser detectados nas urnas eletrônicas \cite{VandeGraaf2017}, tornando-se alvo de críticas de especialistas \cite{Aranha2018} e de discussões populares \cite{Payao2018}.  

Ao mesmo tempo uma revolução começou a se desenhar com base numa nova maneira de representar dados transacionais. \cite{Nakamoto2008} publicou um artigo descrevendo o que passou a se chamar \textit{blockchain}, uma cadeia crescente de blocos de dados, encadeados entre si através de \textit{\glspl{hash1}} (o bloco atual sempre possuí o \textit{\gls{hash1}} do último bloco). Internamente os blocos são formados por listas de dados, que por sua vez estão conectados entre si através de uma árvore de \textit{\glspl{hash1}}. Todos os nós da rede recebem as transações dela e criam seus próprios blocos e cadeias de dados, impedindo assim a inclusão de dados espúrios; ao manter evidências em diversos computadores diferentes, impede-se que uma transação existente em apenas um nó, ou numa minoria deles, seja considerada válida. 

Isso trouxe ao cenário mundial a ideia de introduzir essa tecnologia para resolução de diversos problemas, como cadeia de propriedade de imóveis e automóveis, transações financeiras, etc. Algumas abordagens para solução serão tratadas ao longo deste trabalho.  

A proposta deste trabalho é discutir essas abordagens, um breve histórico, além de introduzir uma nova proposta, o Registro Distribuído de Votação Eletrônica (RDVE), que tenta permitir a coleta e processamento de votos utilizando registro distribuído em blockchain, bem como discutir sua viabilidade sob aspectos como confiabilidade, confidencialidade e auditabilidade. 