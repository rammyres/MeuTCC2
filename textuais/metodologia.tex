\chapter{Metodologia}

O método escolhido para atingimento dos objetivos aqui descritos é a experimentação, através da implementação de uma prova de conceito baseado no modelo proposto e a realização de votações simuladas, utilizando-se esse \textit{software}, procedendo a coleta do produto da execução para apuração e análise utilizando-se uma abordagem qualitativa. Este trabalho abordará sistemicamente o processo de produção e testes, registrando estes processos e seus resultados.  

A solução será implementada para execução em hardware genérico, com a configuração composta especificamente de uma placa SBC (\textit{\gls{sbc1}}, ou computador de placa única) \gls{rpi}, telas sensíveis ao toque com câmera embutida. Além da plataforma será desenvolvido um aplicativo, para o sistema operacional Android, que permitirá a gestão das identificações dos usuários, através de pares de chaves \gls{ecdsa1}, além de permitir a comunicação com o hardware através de \textit{\glspl{qr1}}.  Os produtos das votações serão colhidos em memória \textit{flash} (pendrives ou cartões de memória) e analisados através de \textit{software} automatizado. Eles também terão seus resultados armazenados de forma pública, para inspeção pelos interessados. Por fim o resultado será processado através de uma rede distribuída \gls{p2p}. 

O \textit{software} de coleta e dos nós da rede distribuída serão criados desenvolvidos em Python, o \textit{app} será desenvolvido em Dart.  

Considerando os aspectos subjetivos levantados neste trabalho, referentes a conceitos como confiabilidade, privacidade, etc., nos parece haver a necessidade de uma posição interpretativa quanto ao conjunto de ações essenciais a conclusão das ações descritas nos objetivos, bem como a análise de seus resultados. A partir daí será necessário explorar as possibilidades de utilização do modelo descrito acima, através de experimentação.

De forma resumida: 
\begin{itemize}   

\item Abordagem: qualitativa; 
\item Posição epistemológica: interpretativa;
\item Método de pesquisa: experimental;
\item Finalidade: exploratória;
\item Técnica de coleta: observação direta; e
\item Técnica de análise: análise de conteúdo.
\end{itemize}