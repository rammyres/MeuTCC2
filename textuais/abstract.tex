\begin{resumo}[Abstract]
\begin{otherlanguage*}{english}  
	\vspace{\onelineskip}
	\noindent
		This paper aims to discuss the voting technologies and propose a BEV alternative, the RDVE (\textit{Registro Distribuido de Votação Eletrônica}), as solution to, at least, some of the issues we discuss here, and analyze its results. Voting is part of our daily life throughout the history of all democracies, and following their growth, the difficulties in the voting processes and tallying  grew in size and complexity. Becausue of this, electronic and automated schemas for ballot processing and counting became commonplace. Brazil adopted its Direct-Recording Electronic Voting Scheme in the 1990s, but, due to failures poited by experts and lack of transparency it has been heavily criticized. In 2008 the blockchain technology was introduced, enabling transparent, yet private, systems, drawing attention from many in industries like finance and e-Government, which in turn has been working towards a secure eletronic voting system. 
		As result we intend to assess the RDVE as solution, specialy aiming its requisites: confidenciality, trustworthiness and auditability. 
		\par\textbf{Keywords}: ballot, blockchain, elections, electronic, tallying, voting.
\end{otherlanguage*}
\end{resumo}