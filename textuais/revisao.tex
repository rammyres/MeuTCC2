\chapter{Revisão Bibliográfica}
A tecnologia blockchain foi primeiro descrita por \citeonline{Nakamoto2008}, em seu artigo “\textit{Bitcoin: A Peer-to-Peer Electronic Cash System}”, para descrever um sistema de movimentações financeiras eletrônicas. A ideia essencial seria encadear blocos de transações, ligados entre si através do \textit{hash}, já que cada bloco conteria o \textit{hash} do bloco anterior. As transações seriam assíncronas, propagadas através de uma rede Ponto-a-Ponto, em que cada nó processaria e persistiria a transação em sua própria cadeia, mantendo as evidências de cada transação em uma Arvore de \emph{Hashes} \cite{Merkle1988}, também conhecida como \gls{merkle}, impedindo que uma delas possa ser registrada mais de uma vez. As transações não estariam ligadas a contas, mas a pares de chaves assimétricas, tornando-as virtualmente anônimas. Os nós também precisariam de gasto computacional para computar o \textit{\gls{hash1}} de cada bloco \cite{Nakamoto2008}, utilizando um sistema de \textit{\gls{pow}} \cite{gervais2016}, derivado do algoritmo de prevenção de Negação de Serviço Hashcash \cite{back2002}, desestimulando as tentativas de forjar blocos, já que tal custo seria muito elevado.

A estrutura pensada por Nakamoto estaria protegida contra problemas básicos de redes distribuídas, entre eles a Falha Bizantina \cite{Lamport1982}, que descreve como um nó dispersando informações falsas ou falhas, numa rede distribuída em anel, pode influenciar toda a rede. Outro problema potencialmente resolvível pela proposta, neste caso relacionado aos sistemas financeiros eletrônicos, e bastante relevante nesse contexto, é o problema do gasto duplo \cite{Brands}, que permite um usuário utilizar mais de uma vez um \textit{token} representativo de um valor. A tecnologia de encadeamento de blocos passou a ser conhecida como blockchain e as tecnologias baseadas nela são popularmente conhecidas como Registro Distribuído \cite{Sunyaev2020}.  

Essa abordagem despertou interesse de diversas indústrias, especialmente a financeira \cite{Cahill2020}, fazendo com que muitos acadêmicos também se voltassem para o assunto. Várias análises sucederam a publicação e posterior popularização do Bitcoin e das tecnologias que emergiram a partir dele. Questões tão diversas, que vão a simples cópias do produto original (conhecidas como \textit{Altcoins}) \cite{Nguyen2019}, até a aplicação em cadeias de produção e tributação distribuídas \cite{Choi2019} foram descritas na literatura desde então. 

Em consequência dessa popularidade, questões sobre segurança passaram, também, a serem alvo de constante escrutínio. \citeonline{Ma2020} propuseram uma análise detalhada, após pesquisa em diversas plataformas colaborativas baseadas em blockchain, demonstrando impactos da tecnologia em si, em aspectos relevantes tratados neste trabalho, como privacidade e confiabilidade, utilizando exemplos práticos de tecnologias já implementadas, como registros médicos.  

No mesmo sentido se pronunciaram \citeonline{Zhong2020}, quando analisaram superficialmente os chamados \gls{bcs} (\textit{Blockchain-based crowdsourced systems}), focando nos aspectos mais primordiais do Bitcoin e outras \glspl{criptomoeda}, como segurança e privacidade, da mesma forma que \citeonline{Feng2019}, que especificaram as ameaças mais comuns a serviços baseados na tecnologia, como o comprometimento da privacidade e  disponibilidade da rede (ataques \gls{dos1}), além de relatarem como a implementação correta de um registro distribuído pode contribuir para mitigar tais problemas. 

Considerando as características da blockchain, muitos trabalhos continuaram a seguir, \citeonline{VanLier2017} trata de diversos aspectos metafilosóficos do tema, como as possibilidades de integração entre \glspl{scf} e, obviamente seu uso em sistemas de votação, além da votação interna, para aquisição de consenso. Não demorou para que ideias sobre \textit{\gls{egov}} surgissem e abordassem a utilização dessas tecnologias para permitir o registro de eleições, que de forma geral são conhecidos pela sigla BEV (\textit{blockchain enabled voting}).

\citeonline{Onik2019} descreve em seu trabalho \textit{“Privacy-aware blockchain for personal data sharing and tracking”} mecanismos para registro de informações pessoais, mantendo a privacidade e com mecanismos para controle do compartilhamento dessas informações, baseados nas tecnologias discutidas neste trabalho. No mesmo sentido se pronunciaram \citeonline{Warkentin2020}, ao analisarem a “Tríade da CIA” (i.e. confidencialidade, disponibilidade e integridade) aplicada as tecnologias de blockchain, Noções extraídas da análise também levam em conta a impossibilidade de reversão de atos registrados ou negação de sua autoria, o que tornaria a abordagem relevante para sistemas governamentais, como o exemplo citado no trabalho \cite{Warkentin2020}, no estado de Illinois, onde foram testados diversos usos de registros distribuídos em seu governo. 

Diversos usos para \textit{e-Gov} foram propostos recentemente \cite{Olnes2017}, como modelos de registros de veículos, antecedentes criminais, dentre os diversos usos possíveis. \citeonline{Kshetri2018b} vão mais a fundo, destacando que tais sistemas podem prevenir corrupção e fraudes, em decorrência da publicidade preponderante dos registros e sua forçosa imutabilidade. 

\citeonline{Elisa2018}, em seu artigo “\textit{A framework of blockchain-based secure and privacy-preserving E-government system}” delineia de forma bem detalhada mecanismo de funcionamento de sistemas de \textit{e-Government} baseados em registro distribuído, de forma geral, incluindo todos os serviços de \gls{g2g}, \gls{g2c} e \gls{g2b}. 

Historicamente o TSE \cite{Brasil2014} demonstrou interesse em informatizar o sistema de votação Brasileiro desde a década de 80, tomando medidas para informatização das Urnas durante a segunda metade da década de 90. O modelo brasileiro de votação eletrônica atualmente é adotado em todo o território nacional, para todas as eleições majoritárias e proporcionais.    

O impacto da confiança da população em tecnologias como essa já foi explorando, para o uso em sistemas governamentais complexos, como o sistema de votação brasileiro. \citeonline{Moura2017}, apresentaram a percepção negativa dos atuais sistemas eletrônicos de votação, bem como a visão positiva de um sistema baseado em blockchain.  

As ideias, no geral, dão conta que um sistema de votação BEV seria percebido como mais transparente, confiável e auditável, já que seu resultado, o Registro Distribuído, é público e disponibilizado a todos os nós e clientes ligados a rede.  

\citeonline{Curran2018} cita as dificuldades do dito \textit{E-Voting}, que é o termo genérico para votações eletrônicas que não dependam de maquinário específico, e habitualmente propõem a utilização de meios eletrônicos para apuração de votos online, com o uso de aplicativos e \textit{websites}. Potenciais falhas e abertura a fraudes, além das dificuldades de auditar tais sistemas são citados como principais motivos da rejeição da adoção desse modelo.  

Dificuldades para atrair eleitores, especialmente os mais jovens, mais facilmente atraídos por tecnologia e menos por ambientes burocráticos, causando aumentos sucessivos das abstenções, também são dificuldades trazidas as eleições, tanto tradicionais em papel, quanto as utilizado sistemas de votação eletrônico. Entretanto vê a possibilidade de inclusão de sistemas com blockchain para registro e apuração de votos como um meio de desenvolvimento de \textit{apps} e \textit{websites} suficientemente confiáveis para registro de votações eletrônicos.

Dentre os possíveis resultados citados por \citeonline{Curran2018} também constam possibilidades, julgadas por muitos como necessárias, de auditabilidade, tanto individual (i.e., o eleitor é capaz de verificar seu próprio voto) como publica (i.e., é possível a qualquer pessoa auditar a votação como um todo).  

\citeonline{Abuidris2019} fizeram uma pesquisa sobre uma série de soluções propostas para os conceitos levantados por Curran. Os estudiosos identificaram, pelo menos, seis propostas, dentre elas os sistemas \textit{Follow My Vote}, \textit{Agora} e \textit{Polys}, que tem relativa notoriedade. 

Todos esses sistemas têm, como funções básicas, a persistência dos votos em um Registro Distribuído, entretanto com grande enfoque em soluções que não dependam de pontos de coleta de votos, nem maquinário específico, já que eles serão colhidos através de \textit{apps} ou \textit{websites}. Os autores também fizeram uma análise comparativa a respeito de diversas características das soluções, como auditabilidade, descrita nos trabalhos como “verificabilidade” (tradução livre), tanto individual como pública (descrita como “universal”), e listaram objetivos a serem alcançados em soluções futuras, como um meio de identificar eleitores, sem comprometer o sigilo do voto, tornar as experiências de usuários simples o bastante a fim de tornar o acesso fácil a todos os segmentos da população, mecanismos para garantir a escalabilidade e expansão da base de usuários, além de meios para aprimorar a velocidade para inserção dos votos e do processo de apuração.  

\citeonline{Wang2018}, em seu artigo \textit{Large-scale Election Based On Blockchain}, propuseram a ideia de um sistema de contratos inteligentes, codificados sobre a plataforma de criptomoedas \gls{eth}. A proposta, conforme apresentada utilizaria \gls{homo} e assinatura em anel, a fim de preservar a integridade e anonimato da votação na rede, o que implicaria na preservação da privacidade no sistema; todas as fases, incluindo o registro do eleitor e a apuração dos votos seriam completamente anônimos, o que inviabilizaria um processo de auditoria externa. 

\citeonline{Srivastava2018} apresentaram artigo descrevendo um protocolo de votação usando uma das formas de blockchain, desenvolvido com base numa tecnologia derivada deste, que permite a formação de grafos acíclicos, enfatizando a privacidade.  

As ideias quanto as implementações dos sistemas de votação são diversas, especialmente quando tomamos analogias aos maiores sistemas de uso de registros distribuídos, que são as criptomoedas. 

A ideia mais comum é a de gerar “carteiras” onde cada usuário (eleitor) recebe uma “moeda” e pode “gastá-la” em um voto, como descrito por \citeonline{Kshetri2018a}. Os autores classificam esse ramo de estudo de \textit{Blockchain Enabled Voting}, ou BEV, que possui as aplicações e modelos já levantados anteriormente, entretanto enfocando a tecnologia para diversos tipos de votações, inclusive as que não envolvam cargos eletivos, como plebiscitos e consultas públicas. Uma vantagem, no entanto, enfatizada pelos autores, está na dificuldade ou impossibilidade de criar votos falsos, ou fraudes no próprio registro de votação, o que parece ser um tema comum em todos os trabalhos até aqui descritos. Da mesma forma se posicionam \citeonline{Agbesi2019}, que apresentaram em Copenhagen um protocolo bastante detalhado de votação, também baseado na ideia da criação de carteiras e distribuição de moedas previamente a votação.  

Temores de fraudes costumam ser ideias centrais da literatura do tema, como mencionado por \citeonline{Zhang2020}. Um dos principais propósitos de sistemas com registro distribuído de votos seria dificultar ao máximo o uso de mecanismos fraudulentos de votação e registro, quer pela exploração de falhas nos \textit{software} ou pelo uso de mecanismos sociais. Uma das ideia ventiladas seria a concessão de "recompensas" para bons eleitores e registradores de votos, bem como penalidades para os suspeitos de fraudes. Do mesmo modo se pronunciam \citeonline{Zhou2020}, numa análise preliminar sobre os principais mecanismos de privacidade nos sistemas de votação eletrônica disponíveis, a saber: assinaturas cegas, em anel ou por delegação.  

\citeonline{Li2020} propuseram a combinação das tecnologias aqui descritas com a \gls{iot}, permitindo sistemas de contagem automática de votação, com \textit{software} executando em maquinário de baixo custo, tornando a captura e processamento de votos amplamente acessível e uma votação de grande porte barata; tal posição foi compartilhada em proposta de \citeonline{Krishnamurthy2020}, que também detalharam diversos mecanismos de segurança associados a mecanismos de votação e contagem. 

A performance de sistemas blockchain para votação também foi abordado em trabalhos anteriores. Os tempos de entrada e apuração podem ser fatores preponderantes, entretanto quando ajustados apropriadamente podem propiciar velocidade suficiente para que seu uso possa ser levado em consideração para grandes projetos de BEV \cite{Khan2020}.  

Conforme se conclui, a utilização de BEV foi bastante escrutinada nos últimos anos, em vários aspectos e sob diversos prismas. A concordância das diversas publicações do uso de blockchain para \textit{\gls{egovs}} e, especialmente para votação, parece estar se formando. Sistemas informatizados inovadores, baseados em \textit{hardware} de baixo custo e \textit{software} alicerçado em tecnologias emergentes podem ser capazes de produzir sistemas de votação com publicidade e confidencialidade, termos aparentemente antagônicos, mas necessários a um sistema que atenda a proposta deste trabalho. A combinação dos elementos apresentados podem apresentar respostas a questionamentos históricos da votação eletrônica, como a possibilidade de auditoria e manutenção da privacidade dos votantes, tornando o registro distribuído uma alternativa aparentemente viável e eficaz.  