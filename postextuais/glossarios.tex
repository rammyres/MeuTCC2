\newglossaryentry{dos1}
{
    name=DoS,
    description={
    	É uma tentativa de tornar os recursos de um sistema indisponíveis 
    	para os seus utilizadores, através da requisição constante do mesmo 
	    serviço repetidas vezes, em um curto espaço de tempo
    }
}

\newglossaryentry{sbc1}{
	name={Single Board Computer},
	description={
		referem-se a placas únicas, contendo todos elementos como memória e processador integrados e consistem em um computador funciona
	}
}

\newglossaryentry{rpi}{
	name={Raspberry Pi},
	description={
		é uma placa \emph{Single Board Computer} desenvolvida no Reino Unido pela Raspberry Pi
		Foundation.
	}
}

\newglossaryentry{qr1}{
	name={QR Code},
	description={
		\textit{Quick Response Codes} referem-se a códigos bidimensionais, formados por 
		uma	matriz de quadrados de tamanhos variáveis, que podem ser facilmente
		lidos por maquinas equipadas com câmeras
	}
}

\newglossaryentry{ecdsa1}{
	name={ECDSA},
		description={
			\textit{Elliptic Curve Digital Signature Algorithm} ou algoritmo de assinatura digital em curva elíptica é um algoritmo de assinatura digital, que se baseia na estrutura
		algébrica dessas curvas em campos finitos, permitindo um melhor aproveitamento de
		sistemas caóticos
	}
}

\newglossaryentry{hash1}{
	name={Hash},
	description={
		(ou soma) são mapeamentos de dados de tamanhos variáveis em dados de tamanho fixo,
		através de um algoritmo que transaciona os dados por diversos circuítos, físicos ou virtuais
	},
	plural={hashes}
}

\newglossaryentry{pow}{
	name={Proof of Work},
	description={
		(prova de trabalho) é o protocolo utilizado para deter ataques de negação de serviço
		exigindo trabalho, na forma de tempo de processamento, de alguém que requisita um 
		serviço
	}
}

\newglossaryentry{bcs}{
	name={BCS},
	description={
		\emph{Blockchain-based crowdsourced systems} são sistemas colaborativos, em que os 
		dados recebidos são originados por um conjunto de usuários
	}
}

\newglossaryentry{egov}{
	name={e-Government},
	description={
		Governo Eletrônico consiste no uso das tecnologias da informação — além do conhecimento nos processos internos de governo — e na entrega dos produtos e serviços do Estado para seus próprios
		entes (G2G), seus cidadãos (G2C) bem como à indústria (G2B) e no uso de ferramentas eletrônicas e tecnologias da informação para aproximar governo e cidadãos. 
	}
}


\newglossaryentry{eth}{
	name={Ethereum},
	description={
		é um sistema \textit{open source} de blockchain descentralizado, usado principalmente pela criptomoeda Eth, que contém a capacidade de criar e enforçar contratos inteligentes
	}
}

\newglossaryentry{iot}{
	name={Internet das Coisas},
	description={
		é um conjunto de dispositivos computacionais, eletrônicos (digitais e mecânicos), 
		identificados em uma rede, que tem a capacidade de transferirem dados entre si,
		sem interferência humana
	}
}

\newglossaryentry{p2p}{
	name={P2P},
	description={
		\textit{Peer to peer} ou par a par, é uma rede descentralizada, onde os nós se comunicam entre si livremente, sem a necessidade de nós integradores ou servidores 
	}
}

\newglossaryentry{polis}{
	name={Pólis},
	description={
		era o centro admnistrativo e religioso das cidades gregas antigas 
	}
}

\newglossaryentry{comitia}{
	name={Comitia},
	description={
		literalmente comícios, eram assembléias populares em que os magistrados romanos 
		punham em votações propostas de leis e julgamentos, sem deliberação dos votantes 
		e com aplicação a uma só classe, como o \textit{Consilium plebis}, que criava leis 
		que só se aplicavam aos plebeus 
	}
}

\newglossaryentry{criptomoeda}{
	name={Criptomoeda},
	description={
		é um ativo digital, desenhado para funcionar como meio de troca, onde a propriedade
		desses ativos é registrada em um banco de dados distribuído, formado por diversos computadores em rede P2P, usando criptografia forte para controlar a posse, transferências e criação desses ativos, além de permitir verificar a posse dos mesmos
	}
}

\newglossaryentry{scf}{
	name={Sistema ciber-físico},
	description={
		é um sistema em que os mecanismos de atuação são controlados ou monitorados por algoritmos
	},
	plural={sistemas ciber-físicos}
}

\newglossaryentry{merkle}{
	name={\'{A}rvore de Merkle},
	plural={\'{A}rvores de Merkle},
	description={
		é a estrutura de dados em forma de árvore binária, onde cada nó terminativo (folha)
		contém o \textit{hash} de um bloco de dados e cada nó não-terminativo possui os 
		\textit{hashes} de seus dois nós filhos. O calculo da prova da existência de um dado 
		na árvore tem complexidade $O(\log{}n)$
	}
}
\newglossaryentry{egovs}{
	name={e-Gov},
	description={
		ver \gls{egov}
	}
}
\newglossaryentry{g2g}{
	name={G2G},
	description={
		ver \gls{egov}
	}
}

\newglossaryentry{g2c}{
	name={G2C},
	description={
		ver \gls{egov}
	}
}
\newglossaryentry{g2b}{
	name={G2B},
	description={
		ver \gls{egov}
	}
}

\newglossaryentry{homo}{
	name={Encriptação homomórfica},
	description={
		diz-se dos métodos de encriptação que permitem que os dados cifrados por uma pessoa sejam usadas por uma segunda, sem que a última tenha conhecimento do conteúdo original
	}
}

\newglossaryentry{wallet}{name={Wallet},description={trata-se de um aplicativo que faz o gerenciamento de chaves criptográficas em sistemas \textit{blockchain}}}

\newglossaryentry{qubit}{name={Qubit},plural={Qubits},description={unidade básica da computação quântica, equivalente ao \textit{bit} na computação binária tradicional}}

\newglossaryentry{json}{name={JSON},description={\textit{JavaScript Object Notation}, ou notação de objetos JavaScript, formato de transmissão de dados e de arquivos formados por pares atributo-valor, que são humanamente legíveis}}

\newglossaryentry{dlt}{name={DLT},description={\textit{Distributed Ledger Technology}, ou tecnologias de registro distribuído são tecnolgias derivadas da \textit{blockchain} não utilizados por criptomoedas}}

\newglossaryentry{bev}{name={BEV},description={\textit{Blockchain Enabled Voting}, ou Votaçao Eletrônica Capacitada por Blockchain é o termo cunhado por Kshetri e Voas para designar os sistems de votação eletrônica construidos sobre essa tecnologia}}

\glsaddall
